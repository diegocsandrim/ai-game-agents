\documentclass[a4paper]{article}

%% Language and font encodings
\usepackage[brazil,english]{babel}
\usepackage[utf8]{inputenc}
\usepackage[T1]{fontenc}

%% Sets page size and margins
\usepackage[a4paper,top=3cm,bottom=2cm,left=3cm,right=3cm,marginparwidth=1.75cm]{geometry}

%% Useful packages
\usepackage{amsmath}
\usepackage{graphicx}
\usepackage[colorinlistoftodos]{todonotes}
\usepackage[colorlinks=true, allcolors=blue]{hyperref}

\title{Simulação e jogo com mapas de influência}
\author{Diego Cardozo Sandrim}

\begin{document}
\maketitle

\begin{abstract}
TODO!
\end{abstract}

\section{Introdução}

Jogos de computador e simulações podem utilizar mapas de influência para simular comportamento inteligente de agentes. Mapa de influência é uma matriz de dados que representa a influência de cada agente em um campo.

A matriz de influência é populada com os dados de cada agente que está em um campo. Cada agente pode ter um valor intrínseco de influência, e quanto mais perto do agente o item da matriz está, mais forte é a influência que ele causa naquele item. Somando-se todos os valores de influência dos agentes para uma célula temos a influência completa da matriz.

O valor da influência que um agente causa em função da sua distancia até célula que está sendo calculada pode ser variar para que a influência do agente chegue longe com valores altos, ou caia rapidamente nas células vizinhas, dessa forma podemos das diferentes funções de decaimento de influência para cada agente em campo.

A partir da matriz construída, cada agente pode consulta-la para tomar uma decisão de para onde deve se mover, ou qual ação pode tomar. Múltiplas matrizes de influencia podem ser construídas para que o agente tome diferentes decisões segundo o estado interno do agente.

Matrizes de influência são caras computacionalmente pois tem complexidade de $O(N.M.A)$, onde $N$ é o número de colunas da matriz, $M$ o número de linhas e a o número de agentes. Por esse motivo é comum construir simplificações para calcular a matriz, mesmo que isso tenha impacto na precisão dos dados.

Uma das formas de simplificar o calculo da matriz de influência é limitar o quão distante a influência de um agente pode chegar. Com esse método a complexidade fica em $O(I.A)$, onde $I$ é a quantidade de células onde o agente oferece influência, que deve ser menor que $N.M$. Uma desvantagem desse método é que uma numerosa quantidade de agentes poderia causar uma forte influência a uma certa distancia, mas se essa distancia for menor que o raio usado na simplificação, a influência será desconsiderada.

\section{Aplicações}

No presente trabalho a técnica de mapas de influência foi aplicada a dois sistemas. O primeira se refere a uma simulação de embarque de passageiros em um vagão de trem. O segundo trata-se de um jogo popularmente conhecido como rouba-bandeira. Os dois cenários foram desenvolvidos na linguagem Lua, com o \textit{framework} LÖVE.

\subsection{Simulação de embarque}

\subsubsection{Público alvo}

A simulação de embarque de trem tem como público alvo projetistas de trens. Eles poderiam se beneficiar dessa simulação para testar como as pessoas se comportariam em um modelo de vagão sendo projetado. Os testes podem revelar medidas sobre qual o tempo de embarque, qual percurso as pessoas fazem entre a porta e o local onde se acomodam, onde as pessoas se concentrariam e qual a probabilidade de choque entre passageiros.

\subsubsection{Apresentação}

O cenário se trata de uma plataforma de embarque, com um vagão de trem com as portas abertas. O vagão é formado por suas paredes, espaços abertos representando as portas assentos vazios dentro do vagão. Diversos usuários estão na plataforma. Assim que o jogo começa os usuários iniciam a movimentação. Existem objetos invisíveis que representam a plataforma e os espaços do trem onde os usuários podem ficar em pé.

\subsubsection{Descrição técnica}

O jogo contém alguns controles de teclado para o projetista avaliar melhor o cenário. O botão $p$ pausa e continua o jogo. O botão $d$ habilita ou desabilita o modo de depuração, onde é mostrado o mapa de influência. Quando está pausado o botão $s$ faz a simulação rodar pequenos passos enquanto está pressionado. O botão $r$ reinicia a simulação. Os botões $+$ e $-$ controlam a velocidade da simulação. Além disso há um usuário especial em uma cor diferenciada que o projetista pode controlar por todo o cenário com as setas do teclado.

Cada um dos itens do cenário exerce influência. Assentos e vagas em pé no vagão tem influência numericamente positiva, ou seja, atraem os usuários. Usuários, a plataforma de embarque e as paredes do trem tem influência numericamente negativa, dessa forma afastam os usuários.

A posição onde o vagão e seus componentes se encontra é fixa, os usuários são colocados aleatoriamente fora do vagão. Em cada laço do jogo um novo mapa de influência é calculado, cada um dos usuários consulta o mapa de influência e verifica para onde deve ir. Nenhuma rota é traçada, o usuário sempre escolhe em qual direção seguir consultando as 9 células mais próximas dele, a sua própria célula e mais as suas 8 vizinhas.

Foi utilizado o método de limite do raio de ação da influência. Cada um dos tipos itens da simulação tem valores de influência, raio, velocidade de perca de influência com o espaço. Nenhum algoritmo de detecção de influência foi usado, o mapa de influência foi suficiente para evitar colisões. Isso foi feito colocando valores altos de influencia para os usuários e para as paredes do trem, juntamento com valores altos de perca de influência com a distancia.

A função de influência pela distancia utilizada foi:
\[I_{obj} = \frac{obj.influence}{(distance + 1) ^{obj.influenceDecay}}\]

Cada um dos objetos da simulação podem ter valores diferentes para incluencia e decaimento de influencia, permitindo dessa forma a adaptação do modelo para refletir a realidade observada na plataforma.

\subsubsection{Suposições}

O arquivo \textbf{parameter.lua} contém todas as parametrizações possíveis para o sistema. Nesse arquivo podemos ver as suposições realizadas:

\begin{itemize}
\item Usuário tendem a ficar longe de outros usuários,
\item Usuário prefere ficar sentados do que em pé,
\item Assentos oferecem grande atratividade, porém apenas para quem está nele.
\item Lugares de pé se propagam por muitos espaço, visto que todas as pessoas querem entrar no trem.
\end{itemize}

É importante parametrizar esses valores com dados reais de como as pessoas se comportam. Utilizando vídeos gravados de dentro da plataforma, e comparando o comportamento da simulação com o comportamento real podemos obter dados fiéis do comportamento da pessoa, e dessa forma ser uma ferramenta mais precisa para o projetista.

\subsubsection{Atividades futuras}

Alguns itens podem ser melhorados no jogo

\begin{itemize}
\item Parâmetros reais, extraídos de vídeos com comportamento humano sobre o embarque.
\item Alterar a influência de um objeto uma vez que ele muda de estado.Uma vez que o assento está ocupado a influência que ele causa no mapa deve ser alterada.
\item Tipos de assentos. Adicionar assentos preferenciais às opções de assentos.
\item 
\end{itemize}

\subsection{Jogo rouba-bandeira}

\subsubsection{Público alvo}

A simulação de embarque de trem tem como público alvo projetistas de trens. Eles poderiam se beneficiar dessa simulação para testar como as pessoas se comportariam em um modelo de vagão sendo projetado. Os testes podem revelar medidas sobre qual o tempo de embarque, qual percurso as pessoas fazem entre a porta e o local onde se acomodam, onde as pessoas se concentrariam e qual a probabilidade de choque entre passageiros.

\subsubsection{Apresentação}
Premissas e apresentação de cada um dos dois cenários

\subsubsection{Descrição técnica}
Descrição técnica, métodos utilizados, descrição do sistema

\subsubsection{Suposições}
Elementos que estão "simulados" nesse projeto, considerando que não faremos levantamento de dados

\subsubsection{Atividades futuras}
Atividades futuras, imaginando que o projeto fosse ter continuidade depois da disciplina

First you have to upload the image file from your computer using the upload link the project menu. Then use the includegraphics command to include it in your document. Use the figure environment and the caption command to add a number and a caption to your figure. See the code for Figure \ref{fig:frog} in this section for an example.

\begin{figure}
\centering
\includegraphics[width=0.3\textwidth]{frog.jpg}
\caption{\label{fig:frog}This frog was uploaded via the project menu.}
\end{figure}

\subsection{How to add Comments}

Comments can be added to your project by clicking on the comment icon in the toolbar above. % * <john.hammersley@gmail.com> 2016-07-03T09:54:16.211Z:
%
% Here's an example comment!
%
To reply to a comment, simply click the reply button in the lower right corner of the comment, and you can close them when you're done.

Comments can also be added to the margins of the compiled PDF using the todo command\todo{Here's a comment in the margin!}, as shown in the example on the right. You can also add inline comments:

\todo[inline, color=green!40]{This is an inline comment.}

\subsection{How to add Tables}

Use the table and tabular commands for basic tables --- see Table~\ref{tab:widgets}, for example. 

\begin{table}
\centering
\begin{tabular}{l|r}
Item & Quantity \\\hline
Widgets & 42 \\
Gadgets & 13
\end{tabular}
\caption{\label{tab:widgets}An example table.}
\end{table}

\subsection{How to write Mathematics}

\LaTeX{} is great at typesetting mathematics. Let $X_1, X_2, \ldots, X_n$ be a sequence of independent and identically distributed random variables with $\text{E}[X_i] = \mu$ and $\text{Var}[X_i] = \sigma^2 < \infty$, and let
\[S_n = \frac{X_1 + X_2 + \cdots + X_n}{n}
      = \frac{1}{n}\sum_{i}^{n} X_i\]
denote their mean. Then as $n$ approaches infinity, the random variables $\sqrt{n}(S_n - \mu)$ converge in distribution to a normal $\mathcal{N}(0, \sigma^2)$.


\subsection{How to create Sections and Subsections}

Use section and subsections to organize your document. Simply use the section and subsection buttons in the toolbar to create them, and we'll handle all the formatting and numbering automatically.

\subsection{How to add Lists}

You can make lists with automatic numbering \dots

\begin{enumerate}
\item Like this,
\item and like this.
\end{enumerate}
\dots or bullet points \dots
\begin{itemize}
\item Like this,
\item and like this.
\end{itemize}

\subsection{How to add Citations and a References List}

You can upload a \verb|.bib| file containing your BibTeX entries, created with JabRef; or import your \href{https://www.overleaf.com/blog/184}{Mendeley}, CiteULike or Zotero library as a \verb|.bib| file. You can then cite entries from it, like this: \cite{greenwade93}. Just remember to specify a bibliography style, as well as the filename of the \verb|.bib|.

You can find a \href{https://www.overleaf.com/help/97-how-to-include-a-bibliography-using-bibtex}{video tutorial here} to learn more about BibTeX.

We hope you find Overleaf useful, and please let us know if you have any feedback using the help menu above --- or use the contact form at \url{https://www.overleaf.com/contact}!

\bibliographystyle{alpha}
\bibliography{sample}

\end{document}